\title{PWOP vs SA}

\documentclass{article}
\usepackage{amsmath}
\begin{document}

\section*{Definitions}

\begin{description}

\item $N_e$ = Inbreeding effective population size, a measure of how the average inbreeding coefficient changes from one generation to the next.

\item $N$ = Census size of the population of potential parents.

\item $N_p$ = Number of parents contributing at least one gamete to the next generation.

\item $k$ = number of offspring gametes contributed by a single individual from the parental population.

\item $k_i$ = vector of the $k$ values for $i$ individuals.

\item $S$ = number of observed offspring. Each offspring has two parents, so $S$ = $\sum_{}^{} k_i / 2$

\item $P_{same}$ = the chance that two random gametes selected from the offspring generation are from the same parent. So the chance that two random gametes are identical by descent is $P_{same}/2$.

\end{description}

\section*{Equations}
\begin{description}

\item Crow and Denniston (1988):
\begin{equation} \label{crow_eq}
N_e = \dfrac{\overline{k}_i N - 1}{\overline{k}_i-1+ Var_k/\overline{k}_i} 
\end{equation}

In (\ref{crow_eq}), $k_i$ indexes \textbf{all} possible parents, not just those that successfully contributed to the next generation.

\item Waples (2011), equation 2a:
\begin{equation} \label{waples_eq}
 N_e = \frac{\sum_{}^{} k_i - 1}{\frac{\sum_{}^{} k_i^2}{\sum_{}^{} k_i} -1 } 
\end{equation}

Waples (2011) noted that the Crow and Denniston equation holds even when excluding individuals from the parental population that do not contribute offspring.  They also noted that this also holds for a sample of offspring and can be used to estimate the $N_e$ from a sample of offspring by inferring the $k_i$ vector for the parents contributing to the observed offspring. 

\item Wang (2009), equation 10:

\begin{equation} \label{wang_eq}
\frac{1}{N_e} = \frac{1+3\alpha}{4}(Q_1+Q_2+2Q_3)-\frac{\alpha}{2}(\frac{1}{N_1}+\frac{1}{N_2})
\end{equation}

where $Q_1$, $Q_2$, and $Q_3$ are the probabilities of a pair of offspring being paternal, maternal half-sibs and full-sibs respectively.

assuming no inbreeding, $\alpha$ = 0 and (\ref{wang_eq}) becomes:
\begin{equation} \label{wang_noalpha}
\frac{1}{N_e} = \frac{1}{4}(Q_1+Q_2+2Q_3)
\end{equation}

Wang (2009) equation 8 shows that: 
\begin{equation}
(Q_1+Q_2+2Q_3) = 4*P_{same}
\end{equation}

%\begin{equation}
%\frac{(Q_1+Q_2+Q_3)}{2} = P_{same}
%\end{equation}

Substitution into (\ref{wang_noalpha}) leads to 
\begin{equation} 
N_e = \frac{1}{P_{same}}
\end{equation}
I.e. $N_e$ is equal to the one-generation identity by descent within the the offspring gametes.

\item Given a vector $k_i$, we can also calculate $P_{same}$:

recall $N_p$ = length($k_i$)

\begin{equation}
P_{same} = \frac{1}{N_p(N_p-1)} * \sum_{i=1}^{N_p} (k_i*(k_i-1))
\end{equation}
% 1./(N*(N-1)) * np.sum(k_i*(k_i-1))

%All of the above equations reduce to this same relationship.

\end{description}
\end{document}

